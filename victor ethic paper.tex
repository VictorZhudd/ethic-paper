\documentclass[10pt,twocolumn]{article} 

% use the oxycomps style file
\usepackage{oxycomps}

% read references.bib for the bibtex data
\bibliography{references}

% include metadata in the generated pdf file
\pdfinfo{
    /Title (The Occidental Computer Science Comprehensive Project: Goals, Format, and Advice)
    /Author (Justin Li)
}

% set the title and author information
\title{Ethic paper for senior comp project}
\author{Victor Zhu}
\affiliation{Occidental College}
\email{hzhu@oxy.edu}

\begin{document}

\maketitle

\begin{abstract}
   The video games industry has existed for more than 50 years. In these years of growth, the industry produced many different kinds of video games. Some video games are effective training tools for people to learn different skills but some video games are notorious for promoting violence, discriminatory behaviors, and cultural bias. As a video game producer myself, I must consider what impacts will my game have on my players and society. How would the implementation of my gameplay affect the player morally?  

\end{abstract}

\section{Introduction}

Ethics matters, even in the setting of a video game.  Some might argue that since video games happen in a virtual environment, the player's action has no consequence, however, this is not true. A promise that is made in a work email is made in virtual space, yet it is a real promise, people will expect that promise to be fulfilled. A cyberattack on a computer at the government is a real attack and will lead to criminal charges even incarceration. Using Cheating devices in online video games will lead to an account ban. Virtual events happening in Virtual events have real-world consequences and significant reality.
However, there is no real-world consequence of players’ choices in video games, especially in single-player video games. The Player’s choice only affects the in-game world and vice versa.  As the creator of the world that which the players resided, the responsibility of guiding players’ choices lies on the video game designer. In early video games like the Dragon Quest series, the player is often asked to be the savior of the world, the game only allows the player to do “good” Such as: helping the villager or slaying the evil dragon. After the players have done the good deed of the day, they will get rewarded with in-game currency, in-game items, and experience points for them to level up. However, as the industry grows, people are tired of playing goodie-two-shoes characters, they want something that is more morally ambiguous or even immoral. There are video games that reward immoral behavior: such as the GTA series, Payday series, etc. In these series, players are prompt to do illegal things such as murder, bank-robbing, and stealing, Etc. The games encourage players to do immoral deeds by creating an incentive such as including this kind of behavior in a mission,  an achievement, or rewards players with good in-game items.  The video game designers have the power to control players in their games by creating an incentive. Therefore, the video game designers are able to let the player do good in their games. However, what constitutes as good or standard morals in video games? 




\section{Moral in video games}

 As video games evolve, The background of video games gets more diverse. The morals we value in reality might not apply to the world in certain video games built. In Spores, the player plays as aspiring species that are evolving in an unforgiving world. Should the new creature the player creates adhere to human morals? The answer from the Spores’ developer is no. Even though this game is rated E, there are countless “murders” that happen in the game. Players will starts as micro-organisms and then through devouring other organisms around them, they evolve. This kind of behavior is fawned upon in the human world. A person who kills other people for his or her personal benefit would be prosecuted and incarnated immediately. However, this is not the case in the animal kingdom. The animal in the wild adheres to the law of nature, where the strong eats the weak. Aristotle, the great greek philosopher was one of the earliest and most articulate proponents of this viewpoint. He claimed that all living things had a natural hierarchy. Plants, animals, and humans are all capable of consuming nutrients and growing, but only animals and humans have conscious experiences. This suggests that, despite their inferiority to animals and humans, plants serve the requirements of both animals and humans. Likewise, there are hierarchies within the animal kingdom: The superior animals eat the animal of the lower hierarchy. (Regan and Singer, 1989). This kind of idea justifies the murder of the other organism in Spore. However, the organism that the player plays did not make the choice of killing other organisms for its benefit, the player did.
The player, as a conscious human being, made the choice through their character in the game and therefore ultimately responsible for the actions of their in-game characters.  In Kantian philosophy, a moral agent, such as the player, has free will and is ultimately responsible for their actions. If the player controls their avatar in the game to commit murder then the responsibility for the murder lay with the player. Since eating other organisms in Spore only satisfy the player’s character‘s instinct desires but not the moral law of human being, the player’s action, according to Kant, is essentially unjust.
Ethics on video is a complicated matter, the two greatest philosophers of our history have different answers for one simple educational game that was rated E for everyone. 




\section{the ethic of my Game}

The ethic of my game is also ambiguous. My game is called Nirvana Machinas and it is a combination of Rouglite, card deck collection, and bullet hell. The goal of the game is to travel across the post-apocalyptic wasteland to upload the player’s consciousness to an extraterrestrial server in order to achieve Nirvana.  The player is a cyborg monk, pilgrim, zealot, and ascetic who seeks peace and ascension in the post-apocalyptic wasteland.  In the game, the players are on their journey of pilgrimage, which mirrors the Buddhist ascension process.  The player will go through the wasteland, the nuclear hell, the sea of misery, and finally to the data center where the monk will upload his consciousness to the extraterrestrial server. During his journey 
The monk has to fight enemies such as bandits, heretics, and/or monsters of radiation, etc. 
There are two main ethical concerns about my games. First, the monk will go through combats that will result in the death of the enemies. Second, since I am Appropriating Buddhist culture to my game, it might create a cultural bias among my players. 
Since the game is a rogue-lite game that features combat, the monk will constantly engage in battles that would result in the death of the enemy or the monk. One has to ask, is monk action just? 
The game is set in the post-apocalyptic world, where resources such as food, clean water, etc are scarce. Most of the people roaming the wasteland are either scavengers or bandits and in most situations, they are both. Not only the human held hostility against the monk but also the animals, robots, and even the terrain. They attack the monk purely out of greed and desire to survive. The monk fights back only in self-defense. In philosophy, there is something called casus belli. Casus belli is an act or an event that either provokes or is used to justify a war. One of the most fundamental casus belli is the right to defend oneself against the crime of foreign aggression. The “crime” can take many forms, and in the case of game, it is the constant threat of the outside world. However, in any case, a crime of aggression is committed when the peace of the victim is disrupted. “We know the crime because of our knowledge of the peace it interrupts - not the mere absence of fighting, but peace with rights, a condition of liberty and security that can exist only in the absence of aggression itself. The wrong the aggressor commits is to force men and women to risk their lives for the sake of their rights. It is to confront them with the choice: your rights or (some of) your lives!" (Walzer, 51). As Walzer had said in his essay, the crime initiated when the aggressor strips away the right to live one's life, infringes upon the liberty of victims, and Peace is no longer an option, the only just response is war. As Clausewitz said, “A conqueror is always a lover of peace, (such as Napoleon Bonaparte) he would like to make his entry into our state unopposed, in order to prevent this, we must choose war…” (Walzer, 53). As Clausewitz has asserted, the aggressor will always act as if it were a proponent of peace. A robber would say to his or her victim: if you don't fight back, I will let you live. The robber, the bandits in my game, or a foreign invader all wants their victim to be peaceful, so they could commit their crime easier. Thus, not fighting, a pacifist response, grants the invader precisely what they wished for. Therefore it is evident that it is morally right for the monk in my video game to defend himself against anything in the game that infringes upon his right to freely live and continue his journey.

\section{The culture bias of video game  }
In my game, there are many elements of Buddhism involved. However, it is hard to represent an oriental idea without creating cultural bias. Many western video games are adapting "eastern mysticism," the idea that being from an Asian country connects a person to a spiritual and/or magical force that conveniently moves a plot along. Such as games like Shadow Warriors, and Far Cry 4.  However, despite the massive representation of Asian people in video games, the genuine depiction of Asian cultures in games made in the west or adapted for a western audience has been quite limited. 
According to a Nielsen survey of Caucasians, Hispanics, African-Americans, and Asian-Americans on representation in the video game. Asian-Americans are the most dissatisfied with racial representation in video game characters, In fact, 49 percent of Asian-American respondents are unhappy with racial representation in games. Take one of the earliest video games, Shadow Warrior, for example. 
One early western-developed game featuring an Asian protagonist is the 1997 first-person shooter Shadow Warrior. It has revived recently and becomes a franchise. In this game, The player takes on the role of protagonist Lo Wang who must fight through the demon horde to get what he wants. 
A number of problematic matters existed in the game. First, The protagonist’s name is a dick joke with a slight racial implication. Second One of the selling points of the game was  ‘Lo Wang Speak’. It refers to the broken and heavily accented English spoken by the protagonist Lo Wang.  However, the developer chose to combine elements of Japanese and Chinese culture without any taste or real understanding of either culture. When I play the game, it feels like it mocking the culture I inherited. When the game starts, the player will be greeted by the growling voice of Lo Wang, staring at John William Galt. Any person with a brain would know this is clearly the performance of a white man playing an Asian man. It starts with a stereotypical accent with references to “ancient Chinese secret” and using the “L” instead of “R” Shadow Warrior is a fun game but it also creates and contributes to the Asian stereotype. 
In my game, I have used many Buddhist elements to reference some of the journeys of the monk. The three-level of the games: wasteland, nuclear hell, and the sea of misery are all 
 references to what Buddhists think of life and ascension. In Lotus Sutra, the world we live in is described as hell or a sea of misery. The Buddhists think as long as we live, we will live in the sea of misery. . It is suggested in the Sutra, that being ascended is to pass through the sea of misery. The only way out is to abandon all earthly existence and become budda. At end of my game, the monk will abandon everything he has on the earth and upload his consciousness into an extraterrestrial server. This mirrors the ascension in Buddhism.
 I  also used the extraterrestrial server as the reference to the “pure land” of the Buddhist School of Pure land. In their belief, there exist a pure land that is outside our material world, whoever achieves Nirvana will live there and happily forever after. All my elements of Buddhism are derived from the original text of the Buddhist classic, although it has been translated into a post-apocalyptic sci-fi setting.

\section{Conclusion }
The ethic in video games is ambiguous and also does not contribute to the success of video game. The GTA series trumps every moral fiber in a person, however, it is the most successful game series there is out there. GTA 5 is one of the most profitable video games of all time. The truth is nobody is successful in this world because they are ethical. If people are not ethical in the real world, why should we, the video game developer, bind ourselves with these useless values in a virtual world?   
\section{Refernce}
1.  Walzer, Michael. Just And Unjust Wars. Basic Books, 1977.
2. Regan, Tom ,Singer, Peter (eds.) (1989). Animal Rights and Human Obligations. Cambridge University Press.

\end{document}
